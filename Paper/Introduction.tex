\section{Introduction}
\label{S:Intro}


The automated exploration of unknown environments has become one of the foremost challenges in mobile robotics. For a robot to explore an environment, it must map the environment and concurrently localize itself within the environment.  The framework used to perform this task is known as simultaneous localization and mapping (SLAM) and has been well covered in the literature using a variety of techniques \cite{durrant2006simultaneous,bailey2006simultaneous}.

While SLAM is well known and has a rich history of successes using a single robot, it can often be a slow process due to both constraints on the robot, such as speed and data processing, and lack of redundancy, i.e. robot failure \cite{thrun2001probabilistic,burgard2005coordinated}.  To address the speed of mapping and to add redundancy, coordinated or multi-robot SLAM (MRSLAM) was created.

MRSLAM is as it sounds, SLAM using multiple exploring robots. This approach allows for the partition of the physical search space using different robots, typically decreasing the time it takes to map an area, and increasing the likelihood of full map coverage in the event of robot failure. This temporal exploration parallelism does, however, come at the cost of added complexity. The added complexity of MRSLAM comes in two major components: coordination of exploration using multiple search agents and merging the maps of these agents \cite{fox2006distributed}.  Coordinated exploration consists of planning the groups' search path, most often formulated to cover the most search space in minimum time or to optimize some other mapping cost criteria. The other major component, map merging, is the combination of individual robot's observations and maps into one cohesive global map estimate.

In this paper we focus on the map merging aspect and discuss the encounter based MRSLAM technique of Howard's 2006 paper: ``Multi-Robot Simultaneous Localization and Mapping using Particle Filters \cite{howard2006multi}.’’ This paper is of particular interest as it presents a resource efficient online solution to the MRSLAM map merging problem given unknown initial robot poses. 

<Need figure:>

The paper is structured as follows: In \S\ref{S:Back} some background about MRSLAM is provided, and the importance of \cite{howard2006multi} is presented.  In \S\ref{S:Alg} the algorithm of \cite{howard2006multi} is discussed, and in \S\ref{S:Exp} experimental results are provided for a simulated example and the Albert B data set \cite{Radish}.

